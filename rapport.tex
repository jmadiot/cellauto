\documentclass[a4paper]{article}
\usepackage[utf8]{inputenc}
\usepackage[normalem]{ulem}
\usepackage[french]{babel}
\usepackage{verbatim}
\usepackage{graphicx}


\title{Projet Python : Automates Cellulaires}
\author{Jean-Marie Madiot, Sylvain Dailler}
\date{9 janvier 2009}

\begin{document}
\maketitle{}
\abstract{Le but de ce projet est d'écrire un programme permettant d'implémenter des automates cellulaires.}

\section{Automates génériques}

\paragraph{} Nous avons construit une classe python pouvant gérer les automates cellulaires à une, deux et trois dimensions au niveau du calcul. La grille du jeu est torique. On donne une fonction de règles qui détermine en fonction de l'espace \textit{relatif} à la cellule son état au prochain tour.

\paragraph{} Il s'agit du module {\tt automaton} contenant les classes pour les automates cellulaires à une, deux ou trois dimensions.

\paragraph{} Le constructeur d'un automate nécessite les arguments suivants :

\begin{itemize}
\item{{\tt rules} : fonction de règles.}
\item{{\tt w, h} : dimensions}
\item{{\tt default} : état par défaut (par défaut à 0)}
\item{{\tt disp} : affichage console}
\item{{\tt str\_cell} : annexe pour l'affichage console}
\end{itemize}

\section{Interface}

\paragraph{} Test d'interface et de python en mode console dans un premier temps. C'est le module d'affichage de secours hérité des classes de automaton.

\paragraph{} Choix de Tkinter : facilité et apprentissage d'un module standard donc utile pour créer une application rapidement, par rapport à Gtk, un peu plus difficile mais certes plus complet, et plus joli.

\paragraph{} On utilise le widget {\tt Canvas} de Tkinter et contrairement à toute attente, les objets (lignes, rectangles, ...) se superposent sans s'écraser. La gestion de l'affichage d'un nombre minimal de cellules a aussi permi d'améliorer le temps d'affichage.

\paragraph{} La gestion des évènements (clavier et souris) simplifie l'utilisation.

\begin{itemize}
\item{un clic simple modifie l'état (défaut ou non défaut)}
\item{un clic prolongé permet de conserver l'état modifié sans avoir à cliquer sur tous les pixels}
\item{l'appui sur espace permet de lancer et d'arrêter la simulation}
\item{on peut rentrer un nombre précis d'étapes pour avoir un comportement plus précis. (Par exemple, l'exemple {\tt circus} au bout de 110 itérations)}
\item{on peut exporter la grille de l'automate dans la sortie standard (terminal) en appuyant sur {\tt p}, au format python.}
\end{itemize}

\paragraph{} Nous avons décidé d'utiliser une fenêtre de commande, et une fenêtre d'affichage. C'est un choix ergonomique et conceptuel, de plus plus facile à développer. La fenêtre de commande reste active afin de conserver les options et pouvoir ajuster. On peut également créer plusieurs fenêtres. Un problème parfois engendré est que lors de l'exécution de plusieurs simulations simultanées, l'affichage est retardé par le calcul, et l'image se met à jour plus rarement.

\paragraph{} La fenêtre de commande permet de choisir entre plusieurs automates : le jeu de la vie, le récif corallien avec différentes options, ou encore un des life-likes. L'intérêt est que grâce à la modularité, on peut travailler avec n'importe quel automate cellulaire, quelque soit son rayon.

\section{Le module Conway}

\paragraph{} Le Conway's game of life est largement utilisé pour les exemples d'automates cellulaires. On a donc dédié un module pour améliorer en vitesse cet automate.

\paragraph{} En fait, une des opérations les plus coûteuses utilisées est l'appel de fonction passée en argument grâce à un lambda. Le jeu de la vie optimise l'automate générique dans ce cas particulier et nous bénéficions d'un affichage plus fluide.

\paragraph{} Nous avons ajouté des exemples afin de faire profiter à l'utilisateur d'une partie de l'immense potentialité des automates cellulaires, notamment du jeu de la vie.

\section{Récif corallien}

\paragraph{} Le récif corallien est un exemple d'automate cellulaire modélisant un phénomène réel. Les paramètres pris en compte sont l'âge d'une cellule (pour elle-même), son état, vivant ou squelette, pour ses voisins. Les états des cellules correspondent à l'âge de la cellule et change de couleur en fonction de celui-ci, passant chronologiquement dans les couleurs suivantes bleu marine (rien), vert (naissant), jaune, orange, rouge, violet (agonisant), puis noir (mort).

\paragraph{} À noter qu'il existe un modèle de récif corallien parmi les life-likes proposés. Il est accessible en choisissant {\tt Life-like} puis {\tt Coral}, c'est à dire {\tt 45678/3}.

\section{Life-likes}

\paragraph{} Les automates cellulaires \textit{life-likes} sont des automates cellulaires à deux états (vivant ou mort), deux dimensions et tels que l'état d'une cellule à l'étape suivante ne dépend que de son propre état et du nombre de cellules vivantes parmi les huit adjacentes.

\paragraph{} Pour caractériser les life-likes on peut lister les nombres de cellules correspondant à la survie et à la naissance de la cellule. (Et par quart-exclus, la mort).

\paragraph{} On note ainsi le jeu de la vie : {\tt 23/3}, c'est à dire qu'avec 2 ou 3 voisines vivantes, la cellule survit, et avec 3 voisines vivantes, elle naît. (Qu'elle soit morte ou vive)

\paragraph{} La fenêtre de commande permet de choisir parmi quelques exemples de life-likes, dont une version du récif corallien {\tt coral}.

\section{Optimisation}

\paragraph{} Bien que nous avons évité un certain nombre de pièges de programmation python afin de ne pas perdre un temps absurde, le temps de calcul reste élevé pour de si simples opérations.

\paragraph{} Les problèmes d'affichage font partie intégrante du temps écoulé. Le temps dépensé varie selon le nombre de cellules qui ne sont pas à l'état par défaut, et devient vraiment plus important que le calcul pur quand l'automate est plein.

\paragraph{} Le calcul dans la partie calcul des automates est loin d'être complètement optimisé, mais par exemple, dans le cas des life-likes, on n'as pas besoin d'appel de fonction passée en argument, et il semble que les performances en sont améliorées.

\paragraph{} Le problème reste que les opérations sont élémentaires mais pas vraiment regroupables. On pourrait utiliser le module {\tt ctypes} de python pour gérer les tableaux. L'idéal serait une somme de huit matrices décalées toriquement, pour calculer rapidement le nombre de voisin, mais l'opération de décaler les éléments d'un tableau est une opération coûteuse en C. (a fortiori d'une matrice)

\paragraph{} Ainsi une bonne amélioration des performances pourrait être effectuée en faisant le traitement de l'information par un programme externe ou partiellement externe. L'interface restera alors le principal frein à l'affichage. Peut-être que {\tt canvas} n'est pas adapté à la situation, en particulier pour l'affichage de cellules d'un pixel.

\section*{Conclusion}

Nous avons finalement obtenu de bonnes fonctionnalités et une bonne compréhension d'une interface graphique finalement un peu décevante sous unix. Nous saisissons maintenant l'importance du bon choix de structure de données et l'importance et la difficulté de l'optimisation.

\end{document}




